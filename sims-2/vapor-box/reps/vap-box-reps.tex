\documentclass{tufte-handout}

%\geometry{showframe}% for debugging purposes -- displays the margins

\usepackage{amsmath}
\usepackage{amsfonts}
\usepackage{amsthm}
\usepackage{stmaryrd}
\usepackage{graphicx}
\usepackage{setspace}
\usepackage{fancyhdr}
\usepackage[makeroom]{cancel}
\usepackage{booktabs}
\usepackage{units}
\usepackage{fancyvrb}
\fvset{fontsize=\normalsize}

% \pagestyle{fancyplain}

\DeclareMathAlphabet{\mathpzc}{OT1}{pzc}{m}{it}

%% Autoscaled figures
\newcommand{\incfig}{\centering\includegraphics}
\setkeys{Gin}{width=0.9\linewidth,keepaspectratio}

%% Commonly used macros
\newcommand{\eqr}[1]{Eq.\thinspace(#1)}
\newcommand{\pfrac}[2]{\frac{\partial #1}{\partial #2}}
\newcommand{\pfracc}[2]{\frac{\partial^2 #1}{\partial #2^2}}
\newcommand{\pfraca}[1]{\frac{\partial}{\partial #1}}
\newcommand{\pfracb}[2]{\partial #1/\partial #2}
\newcommand{\pfracbb}[2]{\partial^2 #1/\partial #2^2}
\newcommand{\spfrac}[2]{{\partial_{#1}} {#2}}
\newcommand{\mvec}[1]{\mathbf{#1}}
\newcommand{\gvec}[1]{\boldsymbol{#1}}
\newcommand{\script}[1]{\mathpzc{#1}}
\newcommand{\eep}{\mvec{e}_\phi}
\newcommand{\eer}{\mvec{e}_r}
\newcommand{\eez}{\mvec{e}_z}
\newcommand{\iprod}[2]{\langle{#1}\rangle_{#2}}

%\newcommand{\gcs}{\nabla_{\mvec{x}}}
\newcommand{\gcs}{\nabla}
\newcommand{\gvs}{\nabla_{\mvec{v}}}
\newcommand{\gps}{\nabla_{\mvec{z}}}
\newcommand{\dtv}{\thinspace d^3\mvec{v}}
\newcommand{\dtx}{\thinspace d^3\mvec{x}}

\newtheorem{proposition}{Proposition}
\newtheorem{lemma}{Lemma}
\newtheorem{remark}{Remark}

\newcommand{\nts}[1]{{\color{blue} {#1}}}

%Make the items smaller
\newcommand{\cramplist}{
	\setlength{\itemsep}{0in}
	\setlength{\partopsep}{0in}
	\setlength{\topsep}{0in}}
\newcommand{\cramp}{\setlength{\parskip}{.5\parskip}}
\newcommand{\zapspace}{\topsep=0pt\partopsep=0pt\itemsep=0pt\parskip=0pt}

\title{Premiminary results from fluid simulations of Vapor Box divertor}
%\author{Ammar H. Hakim}%
\date{\today}%

\begin{document}
\maketitle

\begin{abstract}
  \noindent Some \emph{very} preliminary results from invicid Euler
  simulations of vapor box divertor concept are presented.
\end{abstract}

\section{Notes on boundary conditions}

At present, the boundary conditions are applied using ``ghost
cells''. I.e. the vapor quantities in the cell just outside the domain
are held fixed to the values computed from the evaporation formula
which depends on the wall temperature\footnote{P. Browning and
  P.E. Potter, ``AN ASSESSMENT OF THE EXPERIMENTALLY DETERMINED VAPOUR
  PRESSURES OF THE LIQUID ALKALI METALS'', in the \emph{Handbook of
    Thermodynamic and Transport Properties of Alkali Metals},
  R.W. Ohse, Ed., 1985}. In the code, this is implemented as:
\begin{verbatim}
function vaporPressure(Twall)
   return math.exp(26.89-18880/Twall-0.4942*math.log(Twall))
end
\end{verbatim}
Once pressure is determined, the number density is computed assuming
that the evaporated vapor is at the same temperature as the wall
% \footnote{It is possible that these BCs are not quite correct. We have
%   computed fluxes based on drifting Maxwellians also, but these are
%   harder to implement. Eventually, I plan to use the flux BCs (rather
%   than the current ghost BCs) exclusively.}.

\section{Notes on equations and solver}

We are solving the standard invicid Navier-Stokes equations (Euler
equations). These are written in conservation law form (in 1D)
\begin{align}
  \frac{\partial}{\partial{t}}
  \left[
    \begin{matrix}
      \rho \\
      \rho u \\
      \rho v \\
      \rho w \\
      E
    \end{matrix}
  \right]
  +
  \frac{\partial}{\partial{x}}
  \left[
    \begin{matrix}
      \rho u \\
      \rho u^2 + p \\
      \rho uv \\
      \rho uw \\
      (E+p)u
    \end{matrix}
  \right]
  =
  0
\end{align}
Extension to higher dimensions is obvious. Here
\begin{align}
  E = \rho \varepsilon + \frac{1}{2}\rho q^2  
\end{align}
is the total energy and $\varepsilon$ is the internal energy of the
fluid and $q^2=u^2 + v^2 + w^2$. The pressure is given by an equation
of state (EOS) $p=p(\varepsilon, \rho)$. For an ideal gas the EOS is
$p = (\gamma-1)\rho \varepsilon$. The specific enthalpy is defined as
$h = (E+p)/\rho$, and is used in constructing numerical fluxes at cell
interfaces\footnote{For details on this please see
  \url{http://ammar-hakim.org/sj/euler-eigensystem.html}}. I am using
ideal gas EOS for now, but this could be replaced later if
needed\footnote{If we expect a mixture of atomic/diatomic Lithium we
  may want to generate EOS tables and use them instead.}.

To solve these equations I am using a robust, second-order,
shock-capturing finite-volume scheme, which has been benchmarked very
carefully on dozens of difficult problems.\footnote{See
  \url{http://ammar-hakim.org/sj/je/je2/je2-euler-shock.html} for 1D
  shock tests,
  \url{http://ammar-hakim.org/sj/je/je22/je22-euler-2d.html} for 2D
  tests and \url{http://ammar-hakim.org/sj/je/je23/je23-euler-3d.html}
  for 3D tests.}.


% \section{Chain of two boxes}

% To get some insight into the flow I have setup a chain of 2
% boxes. Each box is $0.4\times 0.4$~m and the slot connecting them is
% $10$~cm wide, with a $10$~cm length. The left box walls are held to a
% fixed temperature of $950^o$~C and the right box to $300^o$~C. Note
% that baffles are held to the same temperature as the wall. The
% simulation is run to steady-state. Profiles along the centerline of
% the vapor box are shown below.
% \begin{figure}[ht]%
%   \setkeys{Gin}{width=0.3\linewidth,keepaspectratio}
%   \incfig{s3-two-box-chain-mach.png}
%   \incfig{s3-two-box-chain-numDensity.png}
%   \incfig{s3-two-box-chain-temperature.png}
%   \caption{Mach number, number density and temperature along
%     centerline from a two-box chain.}
% \end{figure}

% The 2D colorplots of these quantities are shown below
% \begin{figure}[ht]%
%   \setkeys{Gin}{width=0.5\linewidth,keepaspectratio}
%   \incfig{s3-two-box-chain_mach_00010.png}
%   \incfig{s3-two-box-chain_numDensity_00010.png}
%   \incfig{s3-two-box-chain_temp_00010.png}
%   \caption{Mach number (top), number density (middle) and temperature
%     (bottom) from a two-box chain simulation.}
% \end{figure}

\section{Chain of five boxes: Lithium covered baffles}

I ran a five-box chain simulation to steady state. Each box is
$0.4\times 0.4$~m and the slot connecting them is $10$~cm wide, with a
$10$~cm length. Each box walls (including baffles) are held at a fixed
temperature, as specified in Rob's paper. The simulation is run to
steady-state. Profiles of various quantities are shown below.
\begin{figure}[ht]%
  \setkeys{Gin}{width=0.75\linewidth,keepaspectratio}
  \incfig{s6-four-box-chain-ln-numDensity.png}
  \caption{Number density from a five-box chain, with Lithium
    covered baffles. See {\tt vapor\-box/s6} for input file.}
\end{figure}
\begin{figure}[ht]%
  \setkeys{Gin}{width=0.75\linewidth,keepaspectratio}
  \incfig{s6-four-box-chain-temperature.png}
  \caption{Temperature from a five-box chain, with Lithium covered
    baffles. Note the temperature drops below the condensation point,
    and hence the Lithium would form droplets and even Lithium
    ``snow''. A true snow-flake divertor! However, the incoming heat
    flux along the LCFS from the tokamak will probably prevent the
    temperature dropping so much. It remains to be seen. See {\tt
      vapor\-box/s6} for input file.}
\end{figure}
\begin{figure}[ht]%
  \setkeys{Gin}{width=0.75\linewidth,keepaspectratio}
  \incfig{s6-four-box-chain-xvel.png}
  \caption{X velocity from a five-box chain, with Lithium covered
    baffles. The flow velocity increases rapidly, however, consistent
    with flow into vacuum, asymptotes to a maximum value, which can be
    determined from conservation of enthalpy along streamlines. See
    {\tt vapor\-box/s6} for input file.}
\end{figure}
\begin{figure}[ht]%
  \setkeys{Gin}{width=0.75\linewidth,keepaspectratio}
  \incfig{s6-four-box-chain-mach.png}
  \caption{Mach number from a five-box chain, with Lithium covered
    baffles. For $x>1$ the Mach number increases linearly (consistent
    with flow into vacuum) as the temperature drops rapidly, even
    though the flow velocity reaches a maximum value as shown in the
    above figure. See {\tt vapor\-box/s6} for input file.}
\end{figure}

\section{Chain of five boxes: Equilibriated (reflecting) baffles}

I ran a five-box chain simulation to steady state. Each box is
$0.4\times 0.4$~m and the slot connecting them is $10$~cm wide, with a
$10$~cm length. Each box wall is held at a fixed temperature, as
specified in Rob's paper. The baffles are assumed to be in equilibrium
with the vapor impinging on them, and are hence treated as perfect
reflectors\footnote{The flux in the normal direction on the baffles is
  set to zero, and the fluid is allowed to slip tangentially,
  consistent with the invicid approximation.}. The simulation is run to
steady-state. Profiles of various quantities are shown below.

The key difference between the Lithium covered and equilibrated
baffles cases is the formation of a standing shock just behind each
baffle. This happens as the vapor reflects off the baffles, increasing
back-pressure and eventually forming a standing shock a few
centimeters behind each baffle.

\begin{figure}[ht]%
  \setkeys{Gin}{width=0.75\linewidth,keepaspectratio}
  \incfig{s7-four-box-chain-ln-numDensity.png}
  \caption{Number density from a five-box chain, reflecting baffle
    case. See {\tt vapor\-box/s7} for input file.}
\end{figure}
\begin{figure}[ht]%
  \setkeys{Gin}{width=0.75\linewidth,keepaspectratio}
  \incfig{s7-four-box-chain-temperature.png}
  \caption{Temperature from a five-box chain, reflecting baffle
    case. Note the temperature drops below the condensation point, and
    hence the Lithium would form droplets and even Lithium ``snow''. A
    true snow-flake divertor! However, the incoming heat flux along
    the LCFS from the tokamak will probably prevent the temperature
    dropping so much. It remains to be seen. See {\tt vapor\-box/s7}
    for input file.}
\end{figure}
\begin{figure}[ht]%
  \setkeys{Gin}{width=0.75\linewidth,keepaspectratio}
  \incfig{s7-four-box-chain-xvel.png}
  \caption{X velocity from a five-box chain, reflecting baffle
    case. The flow velocity increases rapidly, however, consistent
    with flow into vacuum, asymptotes to a maximum value, which can be
    determined from conservation of enthalpy along streamlines. See
    {\tt vapor\-box/s7} for input file.}
\end{figure}
\begin{figure}[ht]%
  \setkeys{Gin}{width=0.75\linewidth,keepaspectratio}
  \incfig{s7-four-box-chain-mach.png}
  \caption{Mach number from a five-box chain, reflecting baffle
    case. For $x>1$ the Mach number increases linearly (consistent
    with flow into vacuum) as the temperature drops rapidly, even
    though the flow velocity reaches a maximum value as shown in the
    above figure. See {\tt vapor\-box/s7} for input file.}
\end{figure}


\end{document}